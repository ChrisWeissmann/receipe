\begin{recipe}
    [ % Optionale Eingaben
        preparationtime = {\unit[ca. 45]{min}},
        % bakingtime={\unit[40]{min}},
        % bakingtemperature={\Fanoven\ \unit[170]{C}},
        portion = \portion{4},
        %calory,
        source = Yvonne | Corn
    ]
    {Schwammal-Erbsen-Sojageschnetzeltes}

    \introduction{}

    \ingredients
    {% Zutatenliste
        2 & große Zwiebeln, gehackt\\
        \unit[400]{g} & Erbsen (TK oder frisch)\\
        \unit[800]{g} & Champignons (frisch)\\
        \unit[100]{g} & Sojageschnetzeltes\\
        \unit[500]{ml} & Gemüsebrühe\\
        \unit[400]{ml} & Sojasahne\\
        \unit[1]{EL} & Sojasoße\\
        \unit[1] {TL} & Salz (je nach Geschmack)\\
        \unit[0,5] {TL} & Pfeffer (je nach Geschmack)\\
        \unit[1]{Prise} & Muskatnuss\\
        \unit[1]{Prise} & Piment\\
        \unit[2] {EL} & Bratöl\\
    }

    \preparation{
        \step Geschnetzeltes in einer Schüssel mit heißer Gemüsebrühe übergießen und ca. 15 Minuten ziehenlassen
        \step Zwiebeln in Würfel schneiden und in einer Pfanne andünsten
        \step Schwammal waschen, in Scheiben schneiden und mit scharf anbraten
        \step Gequollenes Geschnetzeltes mit in die Pfanne geben, anbraten und mit einem Schuss Sojasoße ablöschen
        \step Die Sojasahne beigeben und kurz köcheln lassen
        \step Gewürze hinzugeben und abschmecken
        \step Erbsen unterrühren und 5 - 10 Minuten köcheln lassen
        \step Zum Abschmecken der Soße einen Teelöffel Senf beigeben

    }

    \hint{
        \begin{itemize}
            \item Als Beilage Nudeln, Semmelknödel oder Reis servieren
            \item On top zum Servieren frische oder TK Petersilie 
            \item Alternativ kann auch Erbsengeschnetzeltes verwendet werden
        \end{itemize}
    }

\end{recipe}