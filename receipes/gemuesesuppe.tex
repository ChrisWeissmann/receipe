\begin{recipe}
    [ % Optionale Eingaben
        preparationtime = {\unit[30]{min}},
        % bakingtime={\unit[40]{min}},
        % bakingtemperature={\Fanoven\ \unit[170]{C}},
        portion = \portion{2},
        %calory,
        source = Miriam Kraus
    ]
    {Miriams pürierte Gemüsesuppe}

    \introduction{Die richtige Konsistenz ist nur ein bisschen dünner als Pürree}

    \ingredients
    {% Zutatenliste
        2 & kleine Zwiebeln, halbiert\\
        \unit[400]{g} & Gemüse in Stücken (TK oder frisch)\\
        \unit[200]{g} & Kartoffeln, geachtelt\\
        \unit[10]{g} & Ingwer\\
        \unit[450]{ml} & Wasser\\
        \unit[1,5]{EL} & Mandelmus\\
        \unit [1] {TL} & Salz\\
        \unit [5] {Prisen} & Pfeffer
    }

    \preparation{
        \step Zwiebel, Gemüse, Ingwer, Wasser, Salz, Pfeffer in den Thermomix geben
        \step Mixtopfdeckel schließen
        \step 15 Min auf Stufe 'Varoma' und Kochlöffel kochen
        \step Mandelmus zugeben
        \step Suppe in Schritte pürieren
        \begin{itemize}
            \item 20 Sekunden Stufe 4
            \item 10 Sekunden Stufe 6
            \item 10 Sekunden Stufe 8
            \item 20 Sekunden Stufe 9
        \end{itemize}
        \step abschmecken

    }

    \hint{
        \begin{itemize}
            \item Der Suppe insgesamt 600g Gemüse zufügen, je größer der Kartoffelanteil, desto stärkehaltiger und dickflüßiger (idealerweise Kartoffelanteil zwischen 25 und 50%)
            \item Kartoffelalternative: Blumenkohl
            \item Gemüseoptionen: Brokkoli, Blumenkohl, Suppengemüse
            \item Alternative: 400g Kartoffeln und 200g Lauch
        \end{itemize}
    }

\end{recipe}