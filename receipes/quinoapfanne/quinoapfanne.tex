\begin{recipe}
    [ % Optionale Eingaben
        preparationtime = {\unit[30]{min}},
        % bakingtime={\unit[40]{min}},
        % bakingtemperature={\Fanoven\ \unit[170]{C}},
        portion = \portion{2},
        %calory,
        source = Christian Weissmann
    ]
    {Chris's Quinoapfanne}

    \introduction{Eine vegane Abwandlung einer Quinoa-Hähnchen Pfanne.}

    \ingredients
    {% Zutatenliste
        2 & Knoblauchzehen\\
        \unit[1]{EL} & Honig\\
        \unit[1]{Stück} & Ingwer, Walnussgroß\\
        \unit[2]{EL} & Sojasauce\\
        \unit[200]{g} & Fake Hähnchen\\
        \unit[1]{Bund} & Frühlingszwiebel\\
        \unit[250]{g} & Quinoa\\
        \unit[250]{ml} & Wasser\\
        \unit[200]{ml} & Sojasahne\\
        \unit[3]{m.-große} & Möhren\\
    }

    \preparation{
        \step Ingwer und Knoblauch schälen und fein hacken. 
        Mit Sojasaoße und Honig zu Marinade verrühren.
        \step Fake Hähnchen klein schneiden und einige Stunden oder übernacht in Marinade einlegen.
        \step Möhren schälen und in Scheiben schneiden. 
        Zwiebeln in feine Ringe schneiden. 
        Grün der Zwiebeln für später zur Seite legen.
        \step Fake Hähnchen anbraten und zur Seite stellen.
        \step Weiße der Zwiebeln anbraten und das gewaschene und abgetropfte Quinoa dazugeben. 
        Mit 250ml Wass aufgießen. 
        5 Minuten zugedeckt köcheln lassen
        \step Möhren und Sahne untermischen. 
        Mit Pfeffer, Salz und Sojasoße abschmecken und weitere 10 Minuten - oder bis Quinoa weich ist - köcheln lassen.
        \step Fake Hähnchen und das zuvor geschnittene Grün der Frühlingszwiebel unterheben und kurz heiß werden lassen.
    }

    \hint{
        Eine Packung Fake Hähnchen ist normal genug
    }
    /hint{Mit der Sojasoße muss man nicht sparsam sein
    }

\end{recipe}
