\begin{recipe}
    [ % Optionale Eingaben
        preparationtime = {\unit[30]{min}},
        % bakingtime={\unit[40]{min}},
        % bakingtemperature={\Fanoven\ \unit[170]{C}},
        portion = \portion{2},
        %calory,
        source = Christian Weissmann
    ]
    {Chris's Quinoapfanne}

    \introduction{Eine vegane Abwandlung einer Quinoa-Hähnchen Pfanne.}

    \ingredients
    {% Zutatenliste
        2 & Knoblauchzehen\\
        \unit[1]{EL} & Honig\\
        \unit[1]{Stück} & Ingwer, Walnussgroß\\
        \unit[2]{EL} & Sojasauce\\
        \unit[200]{g} & Fake Hähnchen\\
        \unit[1]{Bund} & Frühlingszwiebel\\
        \unit[250]{g} & Quinoa\\
        \unit[250]{ml} & Wasser\\
        \unit[200]{ml} & Sojasahne\\
        \unit[3]{m.-große} & Möhren\\
    }

    \preparation{
        \step Ingwer und Knoblauch schälen und fein hacken. 
        Mit Sojasauce und Honig zur Marinade verrühren.
        \step Fake Hähnchen klein schneiden und einige Stunden oder übernacht in Marinade einlegen.
        \step Möhren Schälen und in scheiben schneiden. 
        Zwiebeln in Feine Ringe schneiden. 
        Grün der Zwiebeln extra legen.
        \step Fake Hähnchen anbraten und zur seite legen.
        \step Weiße der Zwiebeln anbrachen und das gewaschene und abgetropfte Quinoa dazugeben. 
        Mit 250ml Wass aufgießen. 
        5 Minuten zugedeck kochen lassen
        \step Möhren und Sahen untermischen. 
        Mit Pfeffer, Salz und Sojasoße abschmecken und 10 Minuten Köcheln lassen.
        \step Fake Hähnchen und Zwiebeln Grün unterheben und kurz heiß werden lassen.
    }

    \hint{
        Eine Packung Fake Hähnchen ist normal genug, mit der Sojasoße muss man nicht sparsam sein.
    }

\end{recipe}