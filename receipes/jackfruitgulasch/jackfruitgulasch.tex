\begin{recipe}
    [ % Optionale Eingaben
        preparationtime = {\unit[ca. 60]{min}},
        % bakingtime={\unit[40]{min}},
        % bakingtemperature={\Fanoven\ \unit[170]{C}},
        portion = \portion{4},
        %calory,
        source = Yvonne | Corn
    ]
    {Veganes Gulasch mit Jackfruit}

    \graph
    {
        big=receipes/jackfruitgulasch/Gulasch.jpg
    }
    \ingredients
    {% Zutatenliste
        4 & große Zwiebeln, halbe Ringe\\
        4 & Paprika\\
        2 & Knoblauchzehen\\
        \unit[400]{g} & Jackfruit\\
        \unit[750]{ml} & Gemüsebrühe\\
        \unit[2]{EL} & Mehl\\
        \unit[250]{ml} & Rotwein\\
        \unit[2]{EL} & Tomatenmark\\
        \unit[2] {EL} & Sojasoße\\
        \unit[1] {TL} & Kümmel, gemahlen (je nach Geschmack)\\
        \unit[1]{TL} & Smoked Paprika\\
        \unit[1]{TL} & Salz (je nach Geschmack)\\
        \unit[2] {Prisen} & Pfeffer (je nach Geschmack)\\
        \unit[1] {Schuss} & Balsamico\\
        \unit[2] {EL} & Bratöl\\
    }

    \preparation{
        \step Jackfruit in einer Schüssel mit dem Mehl vermengen
        \step Zwiebeln und Paprika in Streifen schneiden, Knoblauch hacken
        \step Zwiebeln in einem Topf scharf andünsten
        \step Jackfruit und Kümmel dazugeben und mit andünsten
        \step Tomatenmark und Sojasoße beigeben und kurz köcheln lassen, Paprika und Knoblauch dazugeben
        \step Mit der Gemüsebrühe und dem Rotwein ablöschen
        \step Auf niedriger Stufe mit Deckel 40 Minuten köcheln lassen, ab und zu umrühren
        \step Mit Salz, Pfeffer und Smoked Paprika abschmecken

    }

    \hint{
        \begin{itemize}
            \item Serviert wird das Gulasch am besten mit Semmelknödel, wahlweise auch mit Nudeln
            \item Für die alkoholfreie Variante Rotwein mit Gemüsebrühe ersetzen
            \item Entweder Dosenjackfruit oder "trockenes" verwenden
        \end{itemize}
    }

\end{recipe}