\begin{recipe}
    [ % Optionale Eingaben
        preparationtime = {\unit[30]{min}},
        % bakingtime={\unit[40]{min}},
        % bakingtemperature={\Fanoven\ \unit[170]{C}},
        %portion = \portion{2},
        %calory,
        source = Miriam
    ]
    {Mandelmilch \& Mandel-Schoko-Bällchen Basisrezept}

    \introduction{Frische Mandelmilch und Bällchen aus dem Pulp}

    \ingredients
    {% Zutatenliste
        \unit[1/2]{Tasse} & Mandeln, mind 4h eingeweicht\\
        \unit[5]{Tasse} & Wasser\\
        \unit[3] & Datteln, entkernt (optional mit einweichen)\\
        \unit[2]{Prisen} & Salz\\
        \unit[2]{TL} & Vanilie\\
        \unit[1/2]{Tasse} & Mandeln, roh\\
        \unit[1]{Tasse} & Datteln, entkernt\\
        \unit[1/4]{Tasse} & Kakaopulver\\
    }

    \preparation{
        \step Die 4h-eingeweichten Mandeln (1/2 Tasse) mit 3 Datteln, 1 Prise Salz, 1 TL Vanilie und Wasser mixen\\
        \step Mandelmilch von Mandelpulp trennen, Mandelmilch in ein Gefäß füllen und kühlend lagern\\
        \\
        \step rohe Mandeln (1/2 Tasse) und entkernte Datteln (1 Tasse) im Mixer nach eschmack zerkleinern (kernig bis eher fein)\\
        \step Mandelpulp, Prise Salz, 1TL Vanilie und 1/4 Tasse Kakao hinzugeben und vermengen\\
        \step Masse in Bällchen rollen\\
    }

    \hint{
        Wenn keine Zeit zum einweichen der Mandeln besteht, doppelte Menge an Mandeln verwenden\\
        Wenn Bällchen-Teig nicht zusammenklebt, dann entweder mehr Mandelmehl oder Ahornsirup hinzufügen\\
        Mandelbällchen optional in Kakao, Kokosflocken oder Hanfsamen rollen\\
    }

\end{recipe}
