\begin{recipe}
    [ % Optionale Eingaben
        preparationtime = {\unit[30]{min}},
        % bakingtime={\unit[40]{min}},
        % bakingtemperature={\Fanoven\ \unit[170]{C}},
        portion = \portion{12},
        %calory,
        source = Miriam Kraus
    ]
    {Miriam's Hummus}

        %  \graph
        %     {% Bilder
        %         small=Kaesekuchen0,    % kleines Bild
        %         big=Kaesekuchen % grosses  Bild
        %     }


        \ingredients
            {% Zutatenliste
                \unit[2]{Dosen(265g)} & Kichererbsen, ca. Flüßigkeit einer Dose\\
                \unit[2]{Esslöffel} & Tahin\\
                1 & Zitrone\\
                \unit[2]{Teelöffel} & Salz\\
                \unit[1]{Teelöffel} & Koriander, gemahlen\\
                \unit[1]{Teelöffel} & Paprikapulver\\
                \unit[1]{Teelöffel} & Kreuzkümmel\\
                \unit[1]{Teelöffel} & Frish gehackter Knoblauch\\
                \unit[35]{g} & Oliven Öl Bis es sämig ist
            }
            
            
        \preparation{
            \step Alle Zutaten in einen Hochleistungsmixer, wie einen Thermomix, geben
            \step Mixen in Schritten:
                \begin{itemize}
                    \item 30 Sekunden auf Stufe 6
                    \item 30 Sekunden auf Stufe 8
                    \item 15 Sekunden auf Stufe 9
                    \item 15 Sekunden auf Stufe 10
                \end{itemize}
        
        }
        \hint{
        \begin{itemize}
            \item Box zum Abfüllen bereit stellen und Tahini-Löffel darin ablegen
            \item Tahini mit zwei Löffeln entnehmen
            \item Menge an öl kann abhängig von den Kichererbsen variieren, solange zwischen mixen und ölzugabe wechseln bis gewünschte Konsistenz erreicht wurde. Wenn man kein Aquafaba rein machen will, 25g mehr Öl oder Wasser
            \item Ist nicht genug öl beigefügt, dann läuft das Mixerblatt hörbar leer da sich die trockene Menge nach oben bewegt
            \item Am Ende abschmecken und nach Bedarf mehr Gewürze hinzufügen
        \end{itemize}
        }

        \introduction{Hummusbrot ist das beste Brot!}
    \end{recipe}
