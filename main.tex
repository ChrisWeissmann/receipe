\documentclass[a4paper, 11pt]{article}
\usepackage[utf8]{inputenc}
\usepackage[T1]{fontenc}
\usepackage[ngerman, english]{babel}
\usepackage{xcookybooky}[nowarnings]
\usepackage{units}
\usepackage{rotating}
\usepackage{lmodern}
\usepackage{tikz}
\usepackage{graphicx}

\setHeadlines
{%  fuer den Fall das ngerman geladen wurde
    inghead = Zutaten,
    prephead = Zubereitung,
    hinthead = Tipps,
    portionvalue = Portionen,
    continuationhead = Fortsetzung,
    continuationfoot = Fortsetzung auf nächster Seite,
}

\usepackage{hyperref}    % must be the last package
\hypersetup{%
    pdfauthor            = {Christian Weissmann},
    pdftitle             = {Daheim-Rezepte},
    pdfsubject           = {Rezepte},
    pdfkeywords          = {Rezepte},
    pdfstartview         = {FitV},
    pdfview              = {FitH},
    pdfpagemode          = {UseNone}, % Options; UseNone, UseOutlines
    bookmarksopen        = {true},
    pdfpagetransition    = {Glitter},
    colorlinks           = {true},
    linkcolor            = {black},
    urlcolor             = {blue},
    citecolor            = {black},
    filecolor            = {black},
}

\hbadness=10000	% Ignore underfull boxes




\begin{document}

    \title{Ein Rezept Buch}
    \author{Das Kollektiv}
    \maketitle
    \tableofcontents

    \vspace{5em}

    \section{Frühstück}

    \begin{recipe}
    [ % Optionale Eingaben
        preparationtime = {\unit[30]{min}},
        % bakingtime={\unit[40]{min}},
        % bakingtemperature={\Fanoven\ \unit[170]{C}},
        portion = \portion{12},
        %calory,
        source = Miriam Kraus
    ]
    {Miriam's Hummus}

        %  \graph
        %     {% Bilder
        %         small=Kaesekuchen0,    % kleines Bild
        %         big=Kaesekuchen % grosses  Bild
        %     }


        \ingredients
            {% Zutatenliste
                \unit[2]{Dosen(265g)} & Kichererbsen, ca. Flüßigkeit einer Dose\\
                \unit[2]{Esslöffel} & Tahin\\
                1 & Zitrone\\
                \unit[2]{Teelöffel} & Salz\\
                \unit[1]{Teelöffel} & Koriander, gemahlen\\
                \unit[1]{Teelöffel} & Paprikapulver\\
                \unit[1]{Teelöffel} & Kreuzkümmel\\
                \unit[1]{Teelöffel} & Frish gehackter Knoblauch\\
                \unit[35]{g} & Oliven Öl Bis es sämig ist
            }
            
            
        \preparation{
            \step Alle Zutaten in einen Hochleistungsmixer, wie einen Thermomix, geben
            \step Mixen in Schritten:
                \begin{itemize}
                    \item 30 Sekunden auf Stufe 6
                    \item 30 Sekunden auf Stufe 8
                    \item 15 Sekunden auf Stufe 9
                    \item 15 Sekunden auf Stufe 10
                \end{itemize}
        
        }
        \hint{
        \begin{itemize}
            \item Box zum Abfüllen bereit stellen und Tahini-Löffel darin ablegen
            \item Tahini mit zwei Löffeln entnehmen
            \item Menge an öl kann abhängig von den Kichererbsen variieren, solange zwischen mixen und ölzugabe wechseln bis gewünschte Konsistenz erreicht wurde. Wenn man kein Aquafaba rein machen will, 25g mehr Öl oder Wasser
            \item Ist nicht genug öl beigefügt, dann läuft das Mixerblatt hörbar leer da sich die trockene Menge nach oben bewegt
            \item Am Ende abschmecken und nach Bedarf mehr Gewürze hinzufügen
        \end{itemize}
        }

        \introduction{Hummusbrot ist das beste Brot!}
    \end{recipe}

    \begin{recipe}
    [ % Optionale Eingaben
        preparationtime = {\unit[10]{min}},
        % bakingtime={\unit[40]{min}},
        % bakingtemperature={\Fanoven\ \unit[170]{C}},
        portion = \portion{2}, 
        %calory,
        source = Miriam Kraus
    ]
    {Miriam's Pumper Waffeln}
        %  \graph
        %     {% Bilder
        %         small=Kaesekuchen0,    % kleines Bild
        %         big=Kaesekuchen % grosses  Bild
        %     }
    \introduction{Pumper Porigage in Waffel form}

    \ingredients[10]
    {% Zutatenliste
        \unit[300]{g} & Haferflocken\\
        \unit[2]{TL} & Leinsamen\\
        \unit[1.5]{TL} & Backpulver\\
        \unit[150]{g} & Pflanzlicher Jogurt\\
        \unit[450]{g} & Pflanzliche Milch\\
        \unit[24]{g} & Öl\\
        \unit[1]{Prise} & Salz\\
        \unit[1]{TL} & Vanille\\
        \unit[1]{TL} & Zimt\\
        \unit[3]{EL} & Protein Pulver\\
    }

    \preparation{
        \step Alle Zutaten in einem Mixtopf z.B. Thermomix zu einer homogenen Masse mischen.
        \step In Waffeleisen rausbraten.

    }

    \hint{
        Jogurt-löffel von Chucky abschlecken lassen.% 2 Esslöffel  machen ca eine Waffel.
    }


\end{recipe}
    
    \section{Warmes}
    \begin{recipe}
    [ % Optionale Eingaben
        preparationtime = {\unit[30]{min}},
        % bakingtime={\unit[40]{min}},
        % bakingtemperature={\Fanoven\ \unit[170]{C}},
        portion = \portion{2},
        %calory,
        source = Miriam Kraus
    ]
    {Miriams pürierte Gemüsesuppe}

    \introduction{Die richtige Konsistenz ist nur ein bisschen dünner als Pürree}

    \ingredients
    {% Zutatenliste
        2 & kleine Zwiebeln, halbiert\\
        \unit[400]{g} & Gemüse in Stücken (TK oder frisch)\\
        \unit[200]{g} & Kartoffeln, geachtelt\\
        \unit[10]{g} & Ingwer\\
        \unit[400]{ml} & Wasser\\
        \unit[1,5]{EL} & Mandelmus\\
        \unit [2] {TL} & Salz\\
        \unit [5] {Prisen} & Pfeffer
    }

    \preparation{
        \step Zwiebel, Gemüse, Ingwer, Wasser, Salz, Pfeffer in den Thermomix geben
        \step Mixtopfdeckel schließen
        \step 15 Min auf Stufe 'Varoma' und Kochlöffel kochen
        \step Mandelmus zugeben
        \step Suppe in Schritte pürieren
        \begin{itemize}
            \item 20 Sekunden Stufe 4
            \item 10 Sekunden Stufe 6
            \item 10 Sekunden Stufe 8
            \item 20 Sekunden Stufe 9
        \end{itemize}
        \step abschmecken

    }

    \hint{
        \begin{itemize}
            \item Der Suppe insgesamt 600g Gemüse zufügen, je größer der Kartoffelanteil, desto stärkehaltiger und dickflüßiger (idealerweise Kartoffelanteil zwischen 25 und 50%)
            \item Kartoffelalternative: Blumenkohl
            \item Gemüseoptionen: Brokkoli, Blumenkohl, Suppengemüse
            \item Alternative: 400g Kartoffeln und 200g Lauch
        \end{itemize}
    }

\end{recipe}
    \begin{recipe}
    [ % Optionale Eingaben
        preparationtime = {\unit[30]{min}},
        % bakingtime={\unit[40]{min}},
        % bakingtemperature={\Fanoven\ \unit[170]{C}},
        portion = \portion{2},
        %calory,
        source = Christian Weissmann
    ]
    {Chris's Quinoapfanne}

    \introduction{Eine vegane Abwandlung einer Quinoa-Hähnchen Pfanne.}

    \ingredients
    {% Zutatenliste
        2 & Knoblauchzehen\\
        \unit[1]{EL} & Honig\\
        \unit[1]{Stück} & Ingwer, Walnussgroß\\
        \unit[2]{EL} & Sojasauce\\
        \unit[200]{g} & Fake Hähnchen\\
        \unit[1]{Bund} & Frühlingszwiebel\\
        \unit[250]{g} & Quinoa\\
        \unit[250]{ml} & Wasser\\
        \unit[200]{ml} & Sojasahne\\
        \unit[3]{m.-große} & Möhren\\
    }

    \preparation{
        \step Ingwer und Knoblauch schälen und fein hacken. 
        Mit Sojasauce und Honig zur Marinade verrühren.
        \step Fake Hähnchen klein schneiden und einige Stunden oder übernacht in Marinade einlegen.
        \step Möhren Schälen und in scheiben schneiden. 
        Zwiebeln in Feine Ringe schneiden. 
        Grün der Zwiebeln extra legen.
        \step Fake Hähnchen anbraten und zur seite legen.
        \step Weiße der Zwiebeln anbrachen und das gewaschene und abgetropfte Quinoa dazugeben. 
        Mit 250ml Wass aufgießen. 
        5 Minuten zugedeck kochen lassen
        \step Möhren und Sahen untermischen. 
        Mit Pfeffer, Salz und Sojasoße abschmecken und 10 Minuten Köcheln lassen.
        \step Fake Hähnchen und Zwiebeln Grün unterheben und kurz heiß werden lassen.
    }

    \hint{
        Eine Packung Fake Hähnchen ist normal genug, mit der Sojasoße muss man nicht sparsam sein.
    }

\end{recipe}

    \begin{recipe}
    [ % Optionale Eingaben
        preparationtime = {\unit[ca. 45]{min}},
        % bakingtime={\unit[40]{min}},
        % bakingtemperature={\Fanoven\ \unit[170]{C}},
        portion = \portion{4},
        %calory,
        source = Yvonne | Corn
    ]
    {Schwammal-Erbsen-Sojageschnetzeltes}

    \introduction{}

    \ingredients
    {% Zutatenliste
        2 & große Zwiebeln, gehackt\\
        \unit[400]{g} & Erbsen (TK oder frisch)\\
        \unit[800]{g} & Champignons (frisch)\\
        \unit[100]{g} & Sojageschnetzeltes\\
        \unit[500]{ml} & Gemüsebrühe\\
        \unit[400]{ml} & Sojasahne\\
        \unit[1]{EL} & Sojasoße\\
        \unit[1] {TL} & Salz (je nach Geschmack)\\
        \unit[0,5] {TL} & Pfeffer (je nach Geschmack)\\
        \unit[1]{Prise} & Muskatnuss\\
        \unit[1]{Prise} & Piment\\
        \unit[2] {EL} & Bratöl\\
    }

    \preparation{
        \step Geschnetzeltes in einer Schüssel mit heißer Gemüsebrühe übergießen und ca. 15 Minuten ziehenlassen
        \step Zwiebeln in Würfel schneiden und in einer Pfanne andünsten
        \step Schwammal waschen, in Scheiben schneiden und mit scharf anbraten
        \step Gequollenes Geschnetzeltes mit in die Pfanne geben, anbraten und mit einem Schuss Sojasoße ablöschen
        \step Die Sojasahne beigeben und kurz köcheln lassen
        \step Gewürze hinzugeben und abschmecken
        \step Erbsen unterrühren und 5 - 10 Minuten köcheln lassen
        \step Zum Abschmecken der Soße einen Teelöffel Senf beigeben

    }

    \hint{
        \begin{itemize}
            \item Als Beilage Nudeln, Semmelknödel oder Reis servieren
            \item On top zum Servieren frische oder TK Petersilie 
            \item Alternativ kann auch Erbsengeschnetzeltes verwendet werden
        \end{itemize}
    }

\end{recipe}
    


    % \begin{recipe}
    [ % Optionale Eingaben
        preparationtime = {\unit[30]{min}},
        % bakingtime={\unit[40]{min}},
        % bakingtemperature={\Fanoven\ \unit[170]{C}},
        portion = \portion{2},
        %calory,
        source = Christian Weissmann
    ]
    {Das beste Rezept von allen!}

    \introduction{Das hier ist das beste Rezept das je gemacht wurde}

    \ingredients
    {% Zutatenliste
        2 & Knoblauchzehen\\
        \unit[1]{EL} & Honig\\
        \unit[1]{Stück} & Ingwer, Walnussgroß\\
        \unit[2]{EL} & Sojasauce\\
        \unit[200]{g} & Fake Hähnchen\\
        \unit[1]{Bund} & Frühlingszwiebel\\
        \unit[250]{ml} & Wasser\\
        \unit[3]{m.-große} & Möhren\\
    }

    \preparation{
        \step Viel machen! 
        Mit vielen Sachen!
        \step Viel machen! 
        Mit vielen Sachen!
        \step Viel machen! 
        Mit vielen Sachen!
        \step Viel machen! 
        Mit vielen Sachen!
        \step Viel machen! 
        Mit vielen Sachen!
        \step Viel machen! 
        Mit vielen Sachen!
        
        \step Genießen!
    }

    \suggestion{
        Viel hilft viel! Vorallem bei Gewürzen!
    }

    \hint{
        Löffel von Checky abschlecken lassen.
    }

\end{recipe}
\end{document}